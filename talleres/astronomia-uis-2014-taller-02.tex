\documentclass[a4paper,12pt]{article}
\usepackage[spanish]{babel}
\hyphenation{co-rres-pon-dien-te}
%\usepackage[latin1]{inputenc}
\usepackage[utf8]{inputenc}
\usepackage[T1]{fontenc}
\usepackage{graphicx}
\usepackage{amsmath}

\usepackage[pdftex,colorlinks=true, pdfstartview=FitH, linkcolor=blue, citecolor=blue, urlcolor=blue, pdfpagemode=UseOutlines, pdfauthor={H. Asorey}, pdftitle={Astronomía - Taller 02} pdfkeywords={vectores}]{hyperref}
\usepackage[adobe-utopia]{mathdesign}

\hoffset -1.23cm
\textwidth 16.5cm
\voffset -2.0cm
\textheight 26.0cm

\begin{document}
\begin{center}
  {\small{Universidad Industrial de Santander - Escuela de Física}}\\
  {\bf{Astronomía para Poetas (Asorey)}}\\
  \vspace{0.4cm}
  Taller 02: Evolución estelar\\ 2014
\end{center}

\renewcommand{\labelenumi}{\arabic{enumi})}
\renewcommand{\labelenumii}{\arabic{enumii})}

\begin{enumerate}
  \item{\bf{Magnitud aparente}}.

    \begin{enumerate}
      \item Calcule la relación entre el brillo de dos estrellas de magnitudes aparentes $m_1=1.3$ y $m_2=4.9$.
      \item ¿Cuál es la más brillante?
      \item ¿Cuál sería la magnitud aparente de la segunda estrella si fuera 10 veces más brillante que la primera?
    \end{enumerate}

  \item{\bf{Magnitud absoluta}}.

    Las magnitudes aparentes del Sol y de la Luna son: $m_\odot=-26.73$ y $m_{Luna}=-12.6$. Entonces:
    \begin{enumerate}
      \item Verifique que el Sol es 449000 veces más brillante que la Luna.
    \end{enumerate}

  \item{\bf{Estrellas}}

    Calcule la luminosidad de Betelgeuse ($M=-5.6$) y de Rigel ($M=-7.0$),
    sabiendo que la magnitud aparente del Sol es $M=4.83$ y su luminosidad
    $L_\odot = 3.85\times10^{26}$ J seg$^{-1}$.

  \item{\bf{Orión}}
    A partir de los colores de Orión trate de estimar
    visualmente la temperatura superficial ($T$) y la clasificación espectral
    (O,B,A,\ldots) de Rigel, Betelgeuse, Bellatrix y Saiph.

  \item{\bf{Observación astronómica}}

    Durante el invierno, mirando hacia el Este y a media altura antes de la
    medianoche es posible observar la constelación de Scorpio. La estrella más
    brillante (Antares) se encuentra a $184$\,pc de la Tierra. Sabiendo que
    la magnitud aparente es es $m=1.09$ y que tiene el mismo color que
    Betelgeuse, calcule la Luminosidad, la masa y el radio de Antares.

  \item{\bf{Temperatura orbital}}

    Sabiendo que la distancia del Sol al planeta Marte ($R=3400$\,km) es $230$
    millones de kilómetros, calcule la temperatura orbital en Marte. ¿Se
    encuentra dentro de la zona de habilitabilidad Solar?


\end{enumerate}
\end{document}
